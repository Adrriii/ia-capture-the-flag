\section{Description des termes techniques}

\begin{itemize}
    \item Framework: Traduit littéralement, signifie "cadre de travail". Désigne un ensemble cohérent de composants éprouvés et réutilisables (bibliothèques, classes, helpers…) ainsi qu'un ensemble de préconisations pour la conception et le développement d'applications.
    \newline
    \item IA: Intelligence artificielle, consiste à mettre en œuvre un certain nombre de techniques visant à permettre aux machines ou programmes d'imiter une forme d'intelligence réelle.
    \newline
    \item Bot: Un programme autonome, qui peut interagir avec des systèmes ou utilisateurs. Très souvent, on parle d'un programme qui se comporte comme un humain dans les jeux vidéos.
    \newline    
    \item Alpha beta: Algorithme de recherche qui sert à réduire le nombre de noeuds calculés par l'algorithme Minimax dans les jeux. L'algorithme Minimax cherche à trouver la meilleure valeur possible d'un plateau de jeu tout en minimisant la meilleure valeur possible d'un plateau de jeu adverse.\newline


    \item Moteur de jeu: collection de code sous forme de bibliothèques et de frameworks afin de réaliser un jeu.
    \newline
    \item Librairie/Bibliothèque/API: un ensemble de fonctions utilitaires, regroupées et mises à disposition afin de pouvoir être utilisées sans avoir à les réécrire.
    \newline
    \item Behavior Tree: Arbre (structure de données) permettant de contrôler le comportement d'une intelligence artificielle, chaque noeud représente une tâche, décision en réponse aux données fournies, ou une structure de contrôle. Permet de réaliser des comportements très complexes et cela en parcourant depuis la racine de l'arbre tout en suivant les branches adéquates en évaluant différentes conditions jusqu'à la décision finale à adopter.
    \newline
    \item Pathfinding: désigne la recherche de chemin entre deux points , noeuds...
    \newline
    \item A*(A star): Algorithme de pathfinding réputé et peu coûteux permettant de rechercher un chemin entre une position initiale et une position finale. Les solutions trouvées par A* sont optimales.
    \newline
    
    \item Frame: Une image affichée à l'écran.\newline 
    
    \item FPS(Frames per second): Fréquence en hertz à laquelle l'affichage est mis à jour.
    \newline
    \item Pygame: Bibliothèque libre multiplateforme qui permet de faciliter le développement de jeux vidéos en langage de programmation Python.
    \newline
    \item Heuristique: Une heuristique est un raisonnement formalisé de résolution de problèmes (représentable par une computation connue) dont on tient pour plausible mais non pour certain qu’il conduira à la détermination d’une solution satisfaisante du problème.
    \newline
    \item Complexité: Évaluation du temps, ressources, et stockage nécessaires à la réalisation d'un algorithme.
    

\end{itemize}{}